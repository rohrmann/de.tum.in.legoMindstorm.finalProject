\documentclass [a4paper,10pt] {article}
\usepackage[latin1]{inputenc}
\usepackage[ngerman]{babel}
\usepackage[T1]{fontenc}
\usepackage{amsmath, amsthm, amssymb}
\usepackage{latexsym}
\usepackage{graphicx}
\usepackage{a4wide}
\setlength{\parindent}{0mm}
\setlength{\parskip}{3mm}
\usepackage{fancyhdr}
\renewcommand{\labelenumi}{(\alph{enumi})}
\pagestyle{fancy}
\headheight 16pt
\renewcommand{\headrulewidth}{0.4pt}
\renewcommand{\footrulewidth}{0.4pt}
\renewcommand{\familydefault}{\sfdefault}
\lhead{Lego Mindstorms Praktikum}
\chead{Team A}
\rhead{SS 2010}
\begin{document}

\section*{\huge{Vor�berlegungen}}
\section{Aufgabenstellung}
Sokoban ist ein Spiel bei dem mit einer Spielfigur Kisten auf vorgegebene Ziele geschoben werden m�ssen. Kisten k�nnen nur geschoben und nicht gezogen werden. Au�erdem k�nnen nicht mehrere Kisten gleichzeitig bewegt werden. \\ 
In dieser Aufgabe sind Roboter zu konstruieren, die ein Sokoban-Feld zun�chst einlesen und dann l�sen. Zu den Regeln eines normalen Sokoban--Spiels wird hierbei zus�tzlich zu dem Schieber ein Zieher eingesetzt. Au�erdem wird ein Kartierer ben�tigt und ein Bauteil, dass die L�sung des Spiels berechnet. \\
\begin{center}
\includegraphics[width=7cm]{soko.png}	\\
\end{center}
Das Spielfeld besteht hier aus schwarzen Linien. Die Kreuzungen sind durch verschiedene Farben gekennzeichnet. Gr�n steht f�r die Ziele der Kisten, Violett f�r die Startr�ume, Gelb f�r den Startraum des Schiebers und ein Pfeil markiert den Startraum des Ziehers und des Kartierers. Rot wird f�r normale Kreuzungen verwendet. 
\section{Zeitplan}
	\textbf{Beginn:} 10. September 2010 \\
	\textbf{Abgabe:} 27. September 2010
	
	\textbf{Konstruktion der Roboter:} ca. 1 Tag \\
	\textbf{Testen von Sensoren und anderen Komponenten:} ca. 2 Tage \\
	\textbf{Implementierung der zusammenfassenden Navigation:} ca. 2 Tage \\
	\textbf{Implementierung des Kartierers:} ca. 3-4 Tage \\
	\textbf{Implementierung des Planers:} ca. 4-5 Tage \\
	\textbf{Implementierung des Ziehers:} ca. 3 Tage \\
	\textbf{Implementierung des Schiebers:} ca. 3 Tage \\
	\textbf{Testen und Optimierung:} ca. 3-4 Tage	
\section{Aufgabenverteilung}
	\textbf{Teammitglieder:} \\
Andreas Bigontina, Michael Bigontina,  Christoph Bruns,  Maximilian Burger, Sebastian Hagen, Wiebke K�pp,Anastasia Panteloglou, Till Rohrmann 

	\textbf{Organisation:} \\ 
Till Rohrmann, Wiebke K�pp \\
	\textbf{Kartierer:} \\ 
Maximilian Burger, Sebastian Hagen \\
	\textbf{Planer:} \\
Till Rohrmann\\ 
	\textbf{Zieher:} \\
Christoph Bruns, Michael Bigontina \\	
	\textbf{Schieber} \\ 
Andreas Bigontina, Anastasia Panteloglou \\
	\textbf{Fehlerbehandlung:} \\
Wiebke K�pp	
\section{L�sungsansatz}
\begin{center}
\includegraphics[width=12cm]{sol.jpg} \\
\end{center}
	\textbf{Navigation:} \\
	Alle drei Roboter verwenden eine gemeinsame Navigation. Die Roboter werden mit dieser �ber einfache Befehle wie \emph{'Bewege dich nach Norden'} oder \emph{'Bewege dich zum n�chsten Raum'} gesteuert. Au�erdem k�nnen auch Koordinaten angegeben werden, zu denen sich ein Roboter bewegen soll. \\
	Die Navigation wird mithilfe eines Pilot realisiert, der �ber verschiedene Behaviors sicherstellt, dass die Roboter der Linie folgen, von ihr nicht abweichen und das die farbigen R�ume erkannt werden.

	\textbf{Aufbau der Roboter:} \\
Alle drei Roboter besitzen drei Sensoren. Zwei Lichtsensoren werden zum Erkennen der Linie verwendet, sie sind am Roboter vorne knapp �ber dem Boden montiert. Zwischen ihnen ist ein Abstand, der in etwa der Breite der Linie entspricht, so dass die Sensoren im Fahren rechts und links der Linie messen  und erkennen wenn der Roboter sich in die Linie dreht und somit korrigiert werden muss. Zwischen den Lichtsensoren, etwas nach hinten versetzt, wird ein Farbsensor montiert. Die Robotor haben au�erdem
\vspace{-0.67cm}
\begin{figure}[htbp]
\begin{minipage}[h]{0.4\linewidth}
\includegraphics[width=1.0\linewidth]{sensoren.jpg}
\end{minipage}
\begin{minipage}[h]{0.6\linewidth}
die Verwendung eines Bricks, zwei Motoren mit gr��eren R�dern zur Bewegung und zwei kleinere St�tzr�der gemeinsam. Hierbei wird versucht die Roboter m�glichst kompakt zu halten, damit beispielsweise bei Zieher und Schieber eine Drehung auch dann  m�glich ist, wenn Zieher und Schieber sich auf benachbarten Feldern befinden.
\end{minipage}
\end{figure}

	\textbf{Kartierer:} \\
Der Kartierer beginnt an seinem Startpunkt das Feld einzulesen. Er speichert das Feld in einem Graphen, in dessen Knoten Informationen �ber den Typ des jeweiligen Knotens (definiert durch die Farbe) und dessen Nachbarn (im Norden, Osten, S�den und Wesen) gespeichert sind. Der Startpunkt ist hierbei als Punkt (0,0) definiert. Die Ausrichtung zu Beginn wird als Norden definiert. Bewegt sich der Roboter nun nach Norden wird die y--Koordinate inkrementiert, bei Osten wird die x--Koordinate inkrementiert und bei S�den und Westen werden x- und y--Koordinate entsprechend dekrementiert. \\
Der Kartierer f�hrt nun eine Tiefensuche auf dem Feld aus und analysiert jede Kreuzung, die er zuvor noch nicht analysiert hat, indem er sich auf ihr dreht und die Richtungen speichert, die dieser Knoten besitzt. Die auf dem Feld erkannte Farbe wird gleichzeitig gespeichert.\\
Nach dem kartieren schickt der Roboter den gesamten Graphen per Bluetooth an den Computer, der dann die Berechnung der L�sung durchf�hrt.
	
	\textbf{Planer:} 
Da die Leistung eines Bricks voraussichtlich nicht f�r die Berechnung der L�sung ausreicht, wird die Berechnung auf einem Computer durchgef�hrt. 


	\textbf{Schieber:}
Der Schieber 

	\textbf{Zieher:}


\end{document}